\chapter{Rate of return of an Investment}

\section{Discounted Cash Flow Technique}

\begin{definition}
    DCF is a method to measure the \textbf{profitability} of investment projects. Unlike fixed annuities (same payment pattern), DCF allows any pattern of cash inflows (returns) and outflows (costs).
\end{definition}

\begin{comments}
\begin{tabular}{|l|l|l|}
\hline
\textbf{Feature} & \textbf{Annuities} & \textbf{Investments} \\
\hline
Payments & Regular intervals, level payment & May vary in time and amount\\
\hline
Risk & Low/fixed & High/low \\
\hline
Examples & Pensions, loans & Stocks, real estate \\
\hline
\end{tabular}
\end{comments}
\\

\begin{comments}
    There are \textbf{two} DCF measures:
\begin{enumerate}  
    \item Net Present Value (NPV) – Present value of all net cash flows. 
    \item Internal Rate of Return (IRR) – Interest rate that makes NPV = 0. 
\end{enumerate}
\end{comments}







%----------------------------------------
\subsection{Cash flows}

\par Notations with Cash flows:
\begin{itemize}[noitemsep] 
    \item $C_t = \text{contributions/outflows (money invested)}$
    \item $R_t = \text{returns/inflows (money received)}$
    \item $c_t = R_t - C_t = \text{net cash flow at time } t $
\end{itemize}
% $c_t > 0$: Net deposit (inflow)
%     $c_t < 0$: Net withdrawal (outflow)

\begin{comments}
    $c_t > 0$: Net deposit (inflow), 
    $c_t < 0$: Net withdrawal (outflow)
\end{comments}

\begin{example}

\textbf{Project Description:}

A company plans to develop and sell a new product. The cash flows are as follows:

\begin{itemize}
    \item Initial investment of \$80{,}000 at year 0.
    \item Additional investments of \$10{,}000 in years 1, 2, and 3.
    \item A contribution of \$20{,}000 in year 4 to launch the product.
    \item Maintenance costs of \$2{,}000 per year from years 5 to 9.
    \item Returns: \$12{,}000 in year 4, \$30{,}000 in year 5, \$40{,}000 in year 6, \$35{,}000 in year 7, \$25{,}000 in year 8, \$15{,}000 in year 9, and \$8{,}000 in year 10.
\end{itemize}

\textbf{Cash Flow Table:}

\begin{center}
\begin{tabular}{cccc}
\toprule
Year & Contributions & Returns & Net Cash Flow ($c_t$) \\
\midrule
0 & 80{,}000 & 0 & -80{,}000 \\
1 & 10{,}000 & 0 & -10{,}000 \\
2 & 10{,}000 & 0 & -10{,}000 \\
3 & 10{,}000 & 0 & -10{,}000 \\
4 & 20{,}000 & 12{,}000 & -8{,}000 \\
5 & 2{,}000 & 30{,}000 & 28{,}000 \\
6 & 2{,}000 & 40{,}000 & 38{,}000 \\
7 & 2{,}000 & 35{,}000 & 33{,}000 \\
8 & 2{,}000 & 25{,}000 & 23{,}000 \\
9 & 2{,}000 & 15{,}000 & 13{,}000 \\
10 & 0 & 8{,}000 & 8{,}000 \\
\bottomrule
\end{tabular}
\end{center}

\textbf{Net Present Value (NPV):}

Let \( i \) be the interest rate (cost of capital), and let \( v = \frac{1}{1 + i} \). Then the NPV of the project is:

\[
\text{NPV}(i) = \sum_{t=0}^{10} c_t v^t = \sum_{t=0}^{10} \frac{c_t}{(1+i)^t}
\]

Where \( c_t \) is the net cash flow in year \( t \).

\textbf{Interpretation:}
\begin{itemize}
    \item If \( \text{NPV}(i) > 0 \): the investment is profitable.
    \item If \( \text{NPV}(i) = 0 \): the investment breaks even.
    \item If \( \text{NPV}(i) < 0 \): the investment is not worth it.
\end{itemize}
\end{example}






\section{Net Present Value}

\begin{definition}
\textbf{Net present value (NPV)} is the difference between the present value of \textbf{cash inflows} and the present value of \textbf{cash outflows} over a period of time. Choose the investment with the greatest positive NPV.
\end{definition}

\begin{comment}
Goal: Measure whether a project is profitable. 
\end{comment}

\textbf{Formula:}
\[
\text{NPV(i)} = \sum (R_t - C_t) v^t = \sum c_t\cdot v^t
\]


\begin{itemize}
    \item $v^t = \frac{1}{(1+i)^t}$ is the discount factor
    \item $i$ = rate of interest per period = required return of the investment = cost of capital
\end{itemize}

\section{Yield Rate - Internal Rate of Return (IRR)}

\begin{definition}
    \textbf{Yield rate} or IRR is the rate such that the PV of cash inflows is equal to the PV of cash outflows. Choose the investment with the greatest IRR.
\end{definition}

\textbf{Formula:}
\[
\text{IRR} = \text{value of } i \text{ such that } \sum (A_t - L_t) v^t = 0
\]

\textbf{Interpretation:}
\begin{comments}
Yield Rate (IRR) = The interest rate where an investment neither gains nor loses money. 
\begin{itemize}
    \item $\text{NPV} > 0$ : Profit
    \item $\text{NPV} < 0$ : Loss
    \item $\text{NPV} = 0$ : Break-even (Yield rate achieved)
\end{itemize}

Should you still invest when  $\text{NPV} = 0$ ? When  $\text{NPV} = 0$, there is no net gain, but no loss either. Reject the investment if there are better opportunities (i.e. another project with  $\text{NPV} > 0$). Accept it when the Yield rate matched the \textbf{inflation rate}, so your money can keep its real value. 
\end{comments}

\subsection*{Connection Between NPV and IRR}

\begin{itemize}
    \item NPV is a function of the interest rate: $P(i)$
    \item IRR is the rate where $P(i) = 0$
\end{itemize}


\section{Reinvestment}

\subsection{Lump Sum Investment + Interest reinvested}
\begin{definition}
    Intuition: \\
    1. Invest 1 unit of money - \textbf{a lump sum/principal amount} - for n periods at rate $i$. \\
    2. Interest is \textbf{reinvested} at rate $j$. 
\end{definition}

\begin{comments}
    Reinvesting is like planting a tree (investment) and using its seeds (interest) to grow more trees instead of eating them. Over time, you get a forest! 
\end{comments}

\begin{comments}
    Illustration: 
    \begin{enumerate}
        \item Start with 1 unit of money - that principal stays in the account for \textbf{n periods.}
        \item Interest earned per period is 1\$ * $i$ = \$i. The principal gets no compounding effect. 
        \item At the end of year 1,2,...,n-1, we reinvest \$i at each end of the year. This pattern follows an annuity-immediate with \textbf{n payments} of \$i, and rate per period is $j$. 
        \item The \textbf{AV} of that annuity is: i * $s_{\angl{n}j}$. 
        \item Add the principle: Total AV = Principle + Reinvested Interest = 1 + i * $s_{\angl{n}j}$. 
        \end{enumerate}
\end{comments}

\begin{formula} (Lump Sum Reinvestment)
Total Accumulated Value = 1 + i * $s_{\angl{n}j}$
Special case: $i = j$, then AV = $(1+i)^n$
\end{formula}

%-----------------------------------------
\subsection{Annuity + Interest reinvested}








%----------------------------------------
\section{Term structure of Interest rate}

\begin{definition}
    \textbf{Term structure}: short-term and long-term interest rates.
\end{definition}