% If you want to see what things look like for a handout, remove the
% comment from the next line.
\newif\ifHandout%\Handouttrue

%%%%%%%%%%%%%%%%%%%%%%%%%%%%%%%%%%%%%%%%%%%%%%%%%%%%%%%%%%%%%%%%%%%%%%%%%
% Adjust margins and page layout.
\addtolength{\topmargin}{-.5in}
\addtolength{\headsep}{.5in}

\oddsidemargin=.5in
\evensidemargin=.5in
\textwidth=5.5in

%%%%%%%%%%%%%%%%%%%%%%%%%%%%%%%%%%%%%%%%%%%%%%%%%%%%%%%%%%%%%%%%%%%%%%%%%
% Load a bunch of packages that extend the basic capabilities of LaTeX
\usepackage{
amsmath,% AMS basic math stuff
amsthm,% AMS theorem defining stuff
amsfonts,% defines the blackboard bold fonts for \Z, \R, etc
longtable,% used to create the tcproof environment below
verbatim,% allows for verbatim output, and also covering up stuff in comments
xspace,% adds an extra space an the end of some commands
multicol,% allows multicolumn output
tikz,% creates the tikzpicture drawing environment
charter,% changes the default font to charter
framed,% used to color in the TIscreen environment below
}

% the next package defines dingbat fonts.  I use dingbat 212 to make a
% thick arrow somewhat like the "to" arrow that's displayed in TI-83
% programming.  
\PassOptionsToPackage{dvipsnames,table}{xcolor}

\usepackage{pifont}

% Makes section headings and cross references act like links
\usepackage[colorlinks,unicode]{hyperref}
\usepackage{thmtools}
\usepackage{xcolor}  % Load after passing options above
\usepackage{actuarialsymbol}
\usepackage{enumitem}
\usepackage{booktabs}
\usepackage{tcolorbox}  % Comes after xcolor is properly set
\usepackage{float}
\usepackage{array}
\usepackage{booktabs} % For \toprule, \midrule, \bottomrule

%%%%%%%%%%%%%%%%%%%%%%%%%%%%%%%%%%%%%%%%%%%%%%%%%%%%%%%%%%%%%%%%%%%%%%%%%%
% Create simple shortcut commands.  Almost everyone who uses LaTeX
% creates commands like thes.
\newcommand{\R}{\mathbb{R}}
\newcommand{\Q}{\mathbb{Q}}
\newcommand{\Z}{\mathbb{Z}}
\newcommand{\N}{\mathbb{N}}

% For some reason basic TeX/LaTeX doesn't define a an lcm command!  So
% I create one
\DeclareMathOperator{\lcm}{lcm}

% \cl stands for "class"
\newcommand{\cl}[1]{[#1]}

% \st stands for "such that", used in set builder notation like so
% \{x \st x^2 = 4\}
\newcommand{\st}{\mid}

% Make an equals sign with a question mark over it
\newcommand{\eq}{\stackrel{?}{=}}

% The next line difines a divides sign with a question mark
\newcommand{\divq}{\stackrel{?}{|}}

\newcommand{\forwards}{\mbox{``$\Longrightarrow$''}\xspace}
\newcommand{\backwards}{\mbox{``$\Longleftarrow$''}\xspace}

% this is to highlight words that are being defined and enter.
\newcommand{\define}[1]{\textbf{#1}}




%%%%%%%%%%%%%%%%%%%%%%%%%%%%%%%%%%%%%%%%%%%%%%%%%%%%%%%%%%%%%%%%%%%%%%%%%%%%
% Create some slightly more sophisticated custom commands.  New
% symbols, or tweaking old ones.  

% The next two lines define a reasonable looking not divides sign.
\DeclareMathSymbol{\nmid}{\mathrel}{AMSb}{"2D}
\newcommand{\notdiv}{\nmid}

% Make the tilde command wider
\renewcommand{\tilde}{\widetilde}

% the following two commands change the way the footnote symbol is
% made to be old fashioned: *, dagger, etc.  
\renewcommand{\thefootnote}{\fnsymbol{footnote}}

% This is how I create marks in the text showing where we finished on
% a given day.  
\def\endclass#1{\par\noindent\hrulefill\fbox{\tiny This is where we
    ended on #1}\hrulefill\vskip 5pt plus 1pt\par }

% Now I redefine things to produce the headings I want.
\pagestyle{headings}
\makeatletter
% First I redefine the \today command
\edef\today{%
  \the\year/\two@digits{\the\month}/\two@digits{\the\day}}
% Now I put the date in the heading
\renewcommand{\@evenhead}{\emph{Financial Mathematics (SOA Exam)} (v.\today)
    \hfill Hanh Hieu Dao \hfill \thepage}
\renewcommand{\@oddhead}{(version \today) \hfill \thepage}
\makeatother

% Redefine the emptyset symbol so it looks nicer
\DeclareMathSymbol{\varnothing}{\mathord}{AMSb}{"3F}
\renewcommand{\emptyset}{\varnothing}


% If I'm making handouts I (1) sometimes insert a pagebreak so the
% Example starts at the top of the page, and (2) cover up the
% solutions (turn them white!)
\ifHandout
  \newcommand{\handoutpagebreak}{\newpage}
  \newenvironment{solution}
    {\smallskip\par\noindent\emph{Solution: }\color{white}}
    {}
\else
  \newcommand{\handoutpagebreak}{}
  \newenvironment{solution}
  {\smallskip\par\noindent\emph{Solution: }}
  {}
\fi

\setlist{noitemsep, topsep=2pt}  
\setlength{\abovedisplayskip}{1pt} % Space before equations
\setlength{\belowdisplayskip}{2pt}  % Space after equations
\setlength{\abovedisplayshortskip}{2pt} % Space before short equations (e.g., inline-like)
\setlength{\belowdisplayshortskip}{2pt} % Space after short equations


% The next few commands are for simulating the output of a TI screen.  
\colorlet{shadecolor}{gray!35}
\newenvironment{TIscreen} 
{\begin{center}\tt
\renewcommand{\in}[1]{##1\\}
\newcommand{\out}[1]{\mbox{}\hfill##1\\}
\begin{minipage}{2in}\begin{snugshade}}
{\end{snugshade}\end{minipage}\end{center}}  


%%%%%%%%%%%%%%%%%%%%%%%%%%%%%%%%%%%%%%%%%%%%%%%%%%%%%%%%%%%%%%%%%%%%%%%%%%%%
% The next page or two of stuff is devoted to creating and
% manipulating the environments for lemma, proof, example, etc.
% The first part is pretty simple, and almost everyone has commands
% like this in every paper, article, book, etc.  
% 
% We start with things like lemmas, theorems, etc.
\definecolor{Periwinkle}{rgb}{0.8, 0.8, 1}
\definecolor{Violet}{rgb}{0.56, 0, 1}

\colorlet{LightGray}{white!90!Periwinkle}
\colorlet{LightOrange}{orange!15}
\colorlet{LightGreen}{green!15}
\colorlet{LightPurple}{Violet!15}


% Now we create things like definitions, examples, comments, etc.  
\newenvironment{comments}{\par\noindent\ignorespaces}{}
\newenvironment{terminology}{\par\noindent\ignorespaces}{}

\declaretheoremstyle[name=Formula,headpunct={}]{prosty}
\declaretheorem[style=prosty]{formula}
\tcolorboxenvironment{formula}{colback=LightGray}

\declaretheoremstyle[name=,headpunct={}, ]{prosty}
\declaretheorem[style=prosty, numbered=no]{definition}
\tcolorboxenvironment{definition}{colback=LightOrange}

\declaretheoremstyle[name=Example,]{prcpsty}
\declaretheorem[style=prcpsty]{example}
\tcolorboxenvironment{example}{colback=LightGreen}

\declaretheoremstyle[name=Theorem,]{prcpsty}
\declaretheorem[style=prcpsty]{theorem}
\tcolorboxenvironment{theorem}{colback=LightPurple}

%%%%%%%%%%%%%%%%%%%%%%%%%%%%%%%%%%%%%%%%%%%%%%%%%%%%%%%%%%%%%%%%%%%%%%%%%%%%%%
%%%%%%%%%%%%%%%%%%%%%%%%%%%%%%%%%%%%%%%%%%%%%%%%%%%%%%%%%%%%%%%%%%%%%%%%%%%%%%
%%%%%%%%%%%%%%%%%%%%%%%%%%%%%%%%%%%%%%%%%%%%%%%%%%%%%%%%%%%%%%%%%%%%%%%%%%%%%%
%%%%%%%%%%%%%%%%%%%%%%%%%%%%%%%%%%%%%%%%%%%%%%%%%%%%%%%%%%%%%%%%%%%%%%%%%%%%%%
% The stuff that follows is a bit tricky, and is well more than most
% people do with LaTeX.  The basic idea is that I want to be able to
% show only examples, or only definitions, or hide only proofs, etc.  


\makeatletter
% the following key values give the set up for hiding certain
% environemnts. Note that "true" is just a dummy value.  It would work
% equally well with "foo".  Each one of these redefines the outer
% environment for an atom to turn the whole thing into a hidden
% comment.  Note that the environments theorem, example, etc. do not
% have to be written in any special way for hide and show to work.
% They get clobbered by hide.  
  \define@key{hide}{terminology}[true]{\renewenvironment{terminology}{\comment}{\endcomment}}
  \define@key{hide}{proof}[true]{\renewenvironment{proof}{\comment}{\endcomment}}
  \define@key{hide}{solution}[true]{\renewenvironment{lemma}{\comment}{\endcomment}}
  \define@key{hide}{comments}[true]{\renewenvironment{comments}{\comment}{\endcomment}}  
  \define@key{hide}{definition}[true]{\renewenvironment{definition}{\comment}{\endcomment}}
  \define@key{hide}{example}[true]{\renewenvironment{example}{\comment}{\endcomment}}
% Now this command can be given a comma separated list of names to
% hide.  Thus, \HideEnvirons{ example, proof} would hide all examples
% and proofs.
\newcommand{\HideEnvirons}[1]{\setkeys{hide}{#1}}


% The \ShowEnvirons command works like this: if it's argument is foo,
% it redefines the *hide* family key for foo to do nothing.  Then, it
% issues a \HideEnvirons command containing a list of all the atom
% types.  So, everything is hidden *unless* the hide-family key has
% been redefinied.
  \define@key{show}{terminology}[true]{\define@key{hide}{terminology}{}}
  \define@key{show}{proof}[true]{\define@key{hide}{proof}{}}
  \define@key{show}{solution}[true]{\define@key{hide}{solution}{}}
  \define@key{show}{comments}[true]{\define@key{hide}{comments}{}}
  \define@key{show}{definition}[true]{\define@key{hide}{definition}{}}
  \define@key{show}{example}[true]{\define@key{hide}{example}{}}
% This command can be given a comma separated list of names to show,
% and only these will be shown.  Thus,
% \ShowEnvirons{example,definition} will show only examples and definitions.
\newcommand{\ShowEnvirons}[1]
{\setkeys{show}{#1}\HideEnvirons{%
    comments,
    definition,
    example,
    proof,
    terminology, 
    solution
  }}
\makeatother

% Comment out the next line to see the full text
\ShowEnvirons{definition,terminology, proof, example, comments, solution}
%------------------------