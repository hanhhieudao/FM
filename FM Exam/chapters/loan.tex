\chapter{Loan}

\section{Debt instruments}

\begin{definition}
A debt instrument is a contract that requires the borrower to repay principal and usually interest at a future date.
\end{definition}

\begin{comments}
    Examples of \textbf{Debt Instruments:}
    \begin{enumerate}
        \item Bonds (corporate, government)
        \item Loans
    \end{enumerate}
\end{comments}

\section{Outstanding Balance Calculation for Level Payments}    

\textbf{Notation:}
\begin{itemize}
    \item \( I_t \): Interest paid during the \( k \)-th period.
    \item \( P_t \): Principal (capital) repaid during the \( k \)-th period.
    \item \( B_t \): Outstanding balance immediately after the \( t \)-th payment.
    \item \( R_t \): Total payment made during the \( t \)-th period (interest + principal).
\end{itemize}

\begin{center}
\renewcommand{\arraystretch}{1.5}
\begin{tabular}{@{} l c l @{}}
\toprule
\textbf{Type} & \textbf{Formula} & \textbf{Interpretation} \\
\midrule
Prospective   & \( B_t = R \cdot a_{n - t} \) & Present value of remaining (future) level payments \\
Retrospective & \( B_t = R \cdot s_t \)       & Accumulated value of past payments \\
\bottomrule
\end{tabular}
\end{center}

\textbf{Loan Equation of Value:}
You can express the loan’s total value \( L \) using the equation of value:

\[
L = B_0 = R_1 v + R_2 v^2 + \cdots + R_n v^n
\]

\section{Loan Amortization (Level/Non-level payments}
\begin{comments}
    A loan can be interpreted as an annuity with payments made in regular intervals, each payment consists of two parts:
\begin{itemize}
    \item \textbf{Interest on the loan}: A cost for borrowing the money.
    \item \textbf{Principal}: The amount of the loan that you borrowed.
\end{itemize}

Over time, interest portion decreases and principal portion increases. Total payments stay the same. 
\end{comments}

\subsection{Interest portion in period t}
At any given time, the interest due for the next payment period will depend on the outstanding balance before t-th payment. 

$$I_t = i \cdot B_{t-1} = i \cdot R \cdot a_{\angl{n - t + 1}} =  R \cdot ({1 - v^{n - t + 1}})$$

\subsection{Principal portion in period t}
\begin{align*}
P_t &= R - I_t \\
    &= R - R \cdot (1 - v^{n - t + 1} \\
    &= R \cdot v^{n - t + 1} \\
\end{align*}

\begin{comments}
Only for level payments: 
\[
\text{FV of principal } P_t \text{ after } k \text{ periods at interest rate } i = P_t \cdot (1 + i)^k = P_{t+k}
\]

\end{comments}

\subsection{Balance}

\begin{itemize}
    \item \textbf{Interest: }Paid to lender (does \textbf{not} reduce balance) 
    \item \textbf{Principal: }Reduces the loan balance \\
\end{itemize}

\begin{comments}
\begin{align*}
\text{Balance after t-th payment} &=\text{Balance before t-th payment} - \text{Principal repaid at time t} \\
&=B_{t-1} - P_t
\end{align*}
\end{comments}

\begin{comments}
\begin{align*}
\text{Balance after t-th payment} &= \text{Balance at time t with interest} - \text{Payment made at time t} \\
&= B_{t-1}(1+i) - R_t
\end{align*}
\end{comments}

\section{Summary}
\begin{definition}
\centering
    \begin{tabular}{|l|l|}
    \hline
    \textbf{Component} & \textbf{Formula at Period \( t \)} \\ \hline
    Total Payment \( R \) & \( R = 1 \) \\ \hline
    Interest \( I_t \) & \( I_t = 1 - v^{n-t+1} \) \\ \hline
    Principal Repaid \( P_t \) & \( P_t = v^{n-t+1} \) \\ \hline
    \hline
    \end{tabular}

\end{definition}