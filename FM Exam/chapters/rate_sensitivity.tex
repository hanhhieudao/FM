\chapter{Measures of Interest Rate Sensitivity}


\section{Inflation}
\begin{definition}
    Inflation = general rise in prices of goods and services over time. It reduces purchasing power of money.
\end{definition}

\begin{comments}
    Inflation and Interest Rates: inflation and interest rates move together over time. Investors 
    demand higher interest to compensate for future inflation. 
\end{comments}

\begin{formula}
    Let $\pi$ is the inflation rate. 
    \[ 1 + i_{\text{real}} = \frac{1 + i_{\text{nominal}}}{1 + \pi} \]
\end{formula}

%---------------------------
\subsection{Payments grow with inflation}
\subsubsection{Present Value}
\begin{definition}
    Each future payment grows by constant \textbf{inflation rate}. 
\end{definition}

\begin{formula} (PV iwth adjusted payments)
    \[
PV = R\left[\frac{1+r}{1+i} + \left(\frac{1+r}{1+i}\right)^2 + \cdots + \left(\frac{1+r}{1+i}\right)^n\right]
    = R(1+r) \cdot \frac{1 - \left(\frac{1+r}{1+i}\right)^n}{i - r}
\]
We discount the annuity with \textbf{nominal rate} as payments have been adjusted with inflation rate.
\end{formula}
\begin{formula} (PV with ray payments, while inflation exists)
    \[
    PV = R\left[\frac{1}{1+i_0} + \frac{1}{(1+i_0)^2} + \cdots + \frac{1}{(1+i_0)^n}\right] = R \cdot a_{\overline{n}|i_0}
    \]

We discount the annuity with \textbf{real interest rate.}
\end{formula}
    

\section*{Summary}
\begin{center}
\begin{tabular}{|l|l|l|}
\toprule
\textbf{Scenario} & \textbf{Formula} & \textbf{When to Use} \\
\midrule
Payments grow with inflation & 
$PV = R(1+r) \cdot \frac{1 - \left(\frac{1+r}{1+i}\right)^n}{i - r}$ & 
Use nominal rate $i$ \\
\midrule
Payments fixed in real terms & 
$PV = R \cdot a_{\overline{n}|i_0}$ & 
Use real rate $i_0$ \\
\bottomrule
\end{tabular}
\end{center}

\subsubsection{Accumulated Value}
\begin{formula} (AV in nominal dollars (not adjusted for inflation))
    \[ \text{AV} = P(1+i_\text{nominal})^n \]
    This is the raw future value of your investment.
\end{formula}

\begin{formula} (AV adjusted for inflation)
    \[ \text{AV} = P (\frac{1+i}{1+r})^n = P(1+i_\text{real}) \]
    This refects the true purchasing power of your money.
\end{formula}

%----------------------------
\section{The Term structure of Interest Rates and Yield Curves}
\begin{definition}
    Term structure of Interest rate shows how interest rates depend on the time (term) to maturity. 
    Generally, longer-term loans/investment results in higher interest rates. This relationship is visualized 
    in a \textbf{yield curve}. 
\end{definition}

\begin{comments}
    Generally — the longer the investment term, the higher the interest rate, because,...
    \begin{itemize}
        \item More time = more risk (like inflation, uncertainty, default). 
        \item Investors want extra return for locking money up longer. 
        \item So lenders charge higher rates for longer loans.
    \end{itemize}
    Extend the table to a continous graph, where y-axis is \textbf{yield} (interest rates/spot rates
    of risk-free bonds) and x-axis is \textbf{maturity}, we obtain a yield curve. Yield curve can be 
    upward-sloping (rates expected to rise), flat (all terms have same rate), and inverted (short-term > long-term, a signal of recession).  

\end{comments}

\begin{table}[h]
\centering
\begin{tabular}{lc}
\toprule
\textbf{Length of investment (years)} & \textbf{Interest rate (Spot rate)} \\
\midrule
1 year & 3\% \\
2 year & 4\% \\
3 year & 6\% \\
4 year & 7\% \\
\bottomrule
\end{tabular}
\end{table}

\begin{terminology}
    \textbf{Yield Curve} is a graph of \textbf{spot rates}. 
    \begin{itemize}
        \item \textbf{Tern}: The length of time until an investment/loan matures/ends 
        \item Term = duration until you get your money back.
        \item Spot rate = \textbf{current} interest rate you would earn if you invested in a \textbf{risk-free bond} today
        and held it until maturity. 
    \end{itemize}
\end{terminology}

\begin{formula}
    When spot rates $i_t$ vary by year, NPV is 
    \[ \text{NPV} = \sum_{t=0}^{n} \frac{c_t}{(1+i_t)^t}\]
\end{formula}

\begin{terminology}
    \textbf{Forward rate:} the \textbf{interest rate} agreed on today for borrowing or investing money in 
    the future. bk.
\end{terminology}


